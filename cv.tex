\documentclass{tccv}
\usepackage[english]{babel}

\begin{document}

\part{Dame Jovanoski}

\section{Intresserad om}
High-performance computation, distributed computation,
concurrency, NoSQL databases, search engines and programming languages.

\section{Arbetserfarenhet}

\begin{eventlist}

\item{September 2017 -- Nu}
     {Tobii AB, Stockholm}
     {Back-end utveclare}

Som del av Back-end teamet:
Arbetade på plattformen för distribuerad datoranalys för att analysera prestanda för eye tracking algoritmer. Introducerade Elasticsearch, arbetade med olika cachemekanismer och infrastruktur.

\break
August
Som del av Tools teamet:
Worked on a in-house developed tool for image collection written in C++ and various tools for data visualizatteknologierion.
\break

\textbf{Back-end teknologier}: Python, C++ (QT), Javascript, Elasticsearch, Azure och databases (MongoDB)\break

\item{Maj 2015 -- Augusti 2017}
     {Vionlabs AB, Stockholm}
     {Backend utveclare}

Responsible for developing SaaS platform from scratch. Developed backend APIs and infrastructure for scalability (using Rancher).
Worked also on data pipelines using Luigi for processing external logs. \break

\textbf{Back-end teknologier}:  Python, Elasticsearch (search och data aggregation),  Luigi data/task pipelines, RethinkDB, Elasticsearch och Mesos/Rancher.\break

\item{Maj 2014 -- Maj 2015}
     {alaTest  AB, Stockholm}
     {Backend utveclare}

Worked on the SaaS platform as part of the back-end team. Backend APIs and databases, data aggregation project and on the matching project for
transforming data from external sources.\break

\textbf{Back-end teknologier}: Python, Elasticsearch, NLP algorithms and similarity functions, MySQL.\break

\item{Juni 2012 -- Augusti 2012}
     {CERN, Switzerland}
     {Praktik}

Developed applications in the area of computer security under the name "Detect Unpatched Web Applications"

\end{eventlist}

\personal
    []
    {\href{https://www.linkedin.com/in/dame-jovanoski-402a8041/}{Linkedin}}
    {+46 73 744 3568}
    {dame.jovanoski@gmail.com}

\section{Utbildning}

\begin{yearlist}

\item[University American College]{2011 -- 2016}
     {Ms. Computer Science, Software Engineering}
     {Nordmakedonien, Skopje}

\item[European University]{2008 -- 2011}
     {Bs. Computer Science, Software Engineering}
     {Nordmakedonien, Skopje}     

\end{yearlist}

\section{Publikationer och certifiering}

\begin{yearlist}

\item{2019}
    {Elasticsearch Engineer I och II}
    {Officielt kurs}


\item{2016}
     {COLING Japan (\href{https://pdfs.semanticscholar.org/5ea2/7e860d62a6a13f53aa0ce6fddf973e8db74d.pdf}{link})}
     {On the Impact of Seed Words on Sentiment Polarity Lexicon Induction}

\item{2015}
     {RANLP Bulgaria (\href{https://pdfs.semanticscholar.org/929c/dabe46cb82c39635558397d6032378845d03.pdf}{link})}
     {Sentiment Analysis for Macedonian Language}
     
\item{2014}
     {Learning from data (\href{https://s3.amazonaws.com/verify.edx.org/downloads/551392b3d221407cb2794ccc7f71d30d/Certificate.pdf}{link})}
     {edX Course}
     
\item{2012}
   {WAD - Web application detector (\href{https://github.com/CERN-CERT/WAD}{link})}
   {Open Source, part of the summer internship program}

\end{yearlist}

\section{Språk}

\begin{factlist}
\item{Makedonska}{Modersmål nivå}
\item{Engelska}{Avancerad nivå}
\item{Svenska}{Övre medelnivå}
\end{factlist}

\section{Kompetenser som utvecklare}

\begin{factlist}

\item{Avancerad nivå}
     {Python, C++, GoLang, MongoDB, Linux, Docker, Jenkins}

\item{Mellanligande nivå}
     {Elasticsearch, Bash, MySQL, Redis}

\item{Grundläggande nivå}
     {ReactJS}

\end{factlist}

\end{document}
